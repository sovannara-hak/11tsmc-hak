\documentclass[11pt]{article}
\usepackage{times} % assumes new font selection scheme installed
\usepackage{amsmath} % assumes amsmath package installed
\usepackage{amsthm} % assumes amsmath package installed
%\usepackage{amssymb}  % assumes amsmath package installed
\usepackage{wasysym}
\usepackage{color}
\usepackage{algorithmic}
\usepackage{algorithm}
\usepackage{tensor}
\usepackage{leftidx}
\usepackage{subfigure}
\usepackage{supertabular}
\usepackage{placeins}
\newcommand{\mbf}[1]{{\mathbf{#1}}}
\newcommand{\dpartial}[2]{\frac{\partial{#1}}{\partial{#2}}}
\DeclareMathOperator*{\argmin}{arg\,min\,} 


\newcommand{\red}[1]{{\textcolor{red}{#1}}}

\begin{document}
\begin{center}
  {\large "Reverse Control for Humanoid Robot Task Recognition" \\
  \small by Sovannara Hak, Nicolas Mansard, Olivier Stasse and Jean Paul Laumond\\
  \small REGULAR PAPER \\
  ~ \\
  \large Detailed answer to reviewers comments and list of changes\\}
\end{center}


\section{General comments}
We would like to thank all the reviewers for their comments and suggestions.
We have revised	our paper taking into account the reviewers suggestions.
In the following, we will discuss the revisions and address the questions raised by the reviewers.

\section{List of changes}
\begin{itemize}
 \item Fig. 1(a) is correctly stated.
 \item In section III, s is defined as a signal.
 \item The term "articular jacobian" may be confusing, we explicit it
 by using the term "differential link between the error and the configuration of the robot"
 \item Hypotheses are a little bit more discussed.
 \item An example of what parameters x could be is given for eq (8): for an exponential decrease
 behavior, x are the parameters of the exponential.

\end{itemize}

\section{Reply to Referee 1}
\begin{quote}
  \red{The definitions of task and behavior are then described in sec IV.  I would suggest the authors to explain them together with the formulation.  In addition, the description of task and behavior which is clear in the formulation becomes a little bit confusing in sec IV.}
\end{quote}

\begin{quote}
  In section IV, which are the limitations of the proposed strategy? In particular, the authors assume to know” the model of the robot”.  I guess they mean the kinematic model, please clarify. How much can the strategy cope with the model uncertainties
\end{quote}
It is indeed the kinematic model of the robot that is assumed to be known. The kinematic model is
required to compute the jacobian and the nullspace projector that are used in the recognition part
(the jacobian is used to project the movement in the task space, the nullspace projector is used
 to cancel a task).
As a consequence, uncertainties in the kinematic model will lead to wrong jacobian or wrong
nullification of a task. However, the influence
of an uncertain model has not been estimated in that works. 
This influence has to be tested in future works involving human movements.

\begin{quote}
 In addition they assume that “all the tasks that may appear in a motion are known”.  This looks like a strong assumption, which requires further explanation.
  In addition I would suggest the authors to better explain the task pool.
\end{quote}
The recognition method is meant to be used with \emph{predictables} movements.
Unknown tasks are not supposed to happen. This assumption allow the 
recognition problem to be treated as a selection problem in a finite
set.
That finite set is materialized by the task pool.
The use of parallel tasks enlarge the movement expressivity of the movement to be analyzed.

\begin{quote}
 \red{Are there cases in which the strategy has low performance?}
 \end{quote}

\begin{quote}
  In addition, the method is based on the observations coming from the robot.  Are you planning to extend the method to also deal with observations coming from different sources, apart from the same robot? 
\end{quote}
The final discussion mentions the use of the method on human motions.
Some works in biomechanics comforts us in a way that it has been showed
that the redundancy is handled by the nervous system in a similar 
way that roboticians handle redundancy for robots.\\

N. Rezzoug and J. Jacquier-Bret and P. Gorce in Computer Methods, 
A method for estimating three-dimensional human arm movement with two electromagnetic sensors,
in Biomechanics and Biomedical Engineering volume 13 issue 6, December 2010

\begin{quote}
 In section V, the differences between the left and right side in Fig. 3-4-5 are not clear.
\end{quote}
The differences between the left and right side are ambiguous on purpose.
The ambiguity is used to illustrate the precision of the recognition method.
 
\begin{quote}
 Why the threshold for the stop criterion is fixed and how do you select its value.  Please make clear the tuning procedure.
\end{quote}
The threshold is empirically set in order to be above numerical noise
due to the successive projections.

\begin{quote}
 During the discussion of the results the authors mention the value of r such as “high” or “low”. I would suggest the authors to define a range (with discussions) to describe a reasonable (or not) result.
\end{quote}
Caracterizing r as "high" or "low" was a mistake. The value of the residues have
to be compared within each others. At the iteration i, the task fitting which has the lowest 
residue involves the better task to select at that iteration.

\section{Reply to Referee 2}

\end{document}
